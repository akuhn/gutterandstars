%!TEX root = gutter+stars.tex
\chapter{Conclusion}
\label{the conclusion}

In this last chapter we summarize the contributions made by this dissertation and point to directions for future work.

\section{Contributions of the Dissertation}

We set out to address the user needs of software engineers regarding code navigation and code understanding. We argued that development tools need to tap on unconventional information found in the source code in order to support software developers with code orientation clues that would be out of their reach without tool support. 

~

\noindent Our key contributions are the following:

\begin{itemize}
\item We identified (\autoref{the introduction}) for fundamental categories of orientation clues used by developers for code navigation and code understanding: \emph{lexical clues} refer to identifier names and concepts, \emph{social clues} referring to a developer's personal network and to internet communities, \emph{episodic clues} referring to personal first-hand memories of a developer, and \emph{spatial clues} referring to the system's architecture or to source code's on-screen position in the development tool.
\item We introduced \emph{Software Cartography} (\autoref{the chapter on codemap}) an approach to create spatial on-screen visualization of software systems based on non-spatial properties. Software maps are stable over time, embedded in the development environment, and can be shared among teams. We evaluated (\autoref{the chapter on the codemap user study}) the approach in a user study, using a prototype implementation, and showed that it supports code orientation by spatial clues.
\item We investigated (\autoref{the chapter on the MSR user study}) how software engineers find answers to technical questions in a series of user studies. We found that developers typically proceed in two steps, first they narrow down their initial clue to a textual clue, and then they query resources on the internet or local documentation for an answer.
\item We presented various prototype tools that tap on unconventional information found in source code in order to provide software developers with code orientation clues that would otherwise be out of their reach. The prototypes support code orientation by providing developers with: lexical clustering of software systems (\autoref{the chapter on hapax}), summarization of software systems and their history (\autoref{the chapter on LogLR}), a story-telling visualization of past contributions to a software system \autoref{the chapter on chronia}), recommendation of experts for work items (\autoref{the chapter on bug reports}), and assessing the credibility of code search results (\autoref{the chapter on codesearch}).
\end{itemize} 

\section{Future Research Directions}

\begin{description}
\item[User Studies on Developer Needs.]
\item[Spatial Representation in Development Tools.]
\item[Social Media in Software Engineering.]
\item[Example-Centric Code Search.]
\item[]
\end{description}

