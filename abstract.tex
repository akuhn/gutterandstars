%!TEX root = gutter+stars.tex
\chapter*{Abstract}

Despite common belief, software engineers do not spend most time writing code. An approximate 50--90\% of development time is spent on \emph{code orientation}, \ie navigation and understanding of source code. This may include reading of local source code and documentation, searching the internet for code examples and tutorials, but also seeking help of other developers. 

In this dissertation, we are interested in how to support software engineers when they are facing technical questions that involve code navigation and code understanding. 

In order to learn about the \codenavigation needs of software engineers we investigated how developers find answers to technical questions. In a qualitative user study, we found that software engineers use at least four kinds of cognitive clues for \codenavigation: 
	lexical clues, 
	social clues, 
	episodic clues 
	and spatial clues.
While lexical clues are well supported by current tools, developers particularly struggle to resolve social, episodic and spatial clues. 
%
Based on those findings we present a series of tools that provide first-class support for the code orientation clues that developers rely on. Common to the presented tools is that they tap on unconventional information found in the source code (such as identifier names, comments, code ownership and versioning information about a system's evolution) in order to provide developers with code orientation clues that would be out of their reach without tool support.

Among the code orientation strategies used by developers, spatial clues stand out for not having a first-class representation in the ecosystem of source code. Given the potential of spatial clues for code navigation, we investigated how to best represent software systems using spatial visualizations. 
%
We introduce \emph{Software Cartography}, a novel cartographic visualization of software systems that enables code orientation by spatial clues. The \emph{software map} visualizations created by our approach offer a spatial representation of software systems based on lexical information found in the system's source code. 
Software maps are stable over time and can be used by individual developers as well as shared by a team. 
We implemented a prototype tool and evaluated it in a qualitative user study. We found that software maps are most helpful to explore search results and call hierarchies.
