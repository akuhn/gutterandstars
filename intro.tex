%!TEX root = gutter+stars.tex
\chapter{Introduction}

Dispite common believe, software engineers do not spend most time writing code. An approximate 50--80\% of their time is spend on code navigation and understanding \cite{many}. This may include reading of local source  source code and documentation, searching the internet for tutorials and code examples, but als seeking the help of another developer. User studies have found that developers use four kinds of cognitive clues for \codenavigation \cite{many}, \ie lexical clues, social clues, episodic clues and spatial clues. Tool support for following-up upon these clues, however, is typically limited to text file processing or hyperlinking of code elements at best. 

In our research, we explore how to support \codenavigation by leveraging development tools with cognitive clues for software orientation. In a user study, we looked at how developers find answers to technical questions in order to learn about \codenavigation strategies. We found that developers typically use lexical and social clues to find answers, however we identified potential for the use of temporal and in particular spatial clues. Thus, a key contribution of our work is ``Software Cartography'' a novel cartographic visualization of software systems to support code orientation by spatial clues. Code maps, as well call them, are a stable and spatial representation of software system based on lexical information found in the system's source code. Code maps can be used by both individual developers and shared in a team. We evaluated our approach in a qualitative user study, with both professional and student users, and found that it is most helpful for exploring search results and call hierarchies.

\section{Types of Code Navigation}

In the context of this work, we are interested in better supporting software engineers when they are facing a technical question that is related to source code. We introduce the term \emph{\codenavigation{}} in order to refer to all programming activities that have in common that software engineers do need to find an answer to a technical question that is related to source code. Code orientation is thus am umbrella term of code navigation and code understanding. 

In the following we categorize \codenavigation according to the kind of cognitive clues being followed-up (lexical clues, social clues, episodic clues and spatial clues) and further characterize \navigation according to its purpose (refind, discovery and learning) and its reach (ranging from current working set to the entire internet).

On the way from a technical question to an answer, developers follow-up different kinds of cognitive clues in order to find to an answer. Examples of these clues are 
	(a) recalling a name or having a guess about the answer's name, or
	(b) knowing a person with expertise or a place on the internet where people share their expertise, or 
	(c) an episodic memory about a past encounter with the answer, for example about having seen it on a mailing list, and 
	(d) sometimes developers even follow-up a spatial clues, like recalling that a given functionality is located at the end of that one specific source text file. 
	
In interviews with developers we found that finding to an answer is typically split in two parts: the first and hard part is boiling the often fuzzy problem description down to a single lexical clue, and in the second part that lexical clue is used to search documentation, tutorials and mailing list for concrete help. Current tool support for \codenavigation is typically limited to \navigation by lexical clues. The above behavior might be an artifact of current tool support \ie that developers aim to find a lexical clue first because they are most easy to follow up, or it might not. In either case, developers do need better tool support to follow-up on social, episodic and spatial clues.
	
\subsection*{Purpose and Reach of Code Orientation}
	
- I wanna refind a certain piece of source code, I wanna refind a certain functionality, I wanna find a solution strategy and learn about an API, I wanna copy paste. If you go back in time, that you know its there, I've learned about it in the past, or you have to find it for the first time. Also a difference in finding code and functionality, like if you have never seen some piece of source code it is maybe easier to recall the person that has told you about it rather than the source code's name. Some if you are looking forward, learning about an API is different than copy pasting some piece of code. Well, sometimes you don't know in the begin with which one you'll gonna end up.
- What's the places that you look at
- the current working set
- your local codebase (which can also have different level, like current project and entire codebase)
- complete codebase of your company
- go to the codebase of the internet, two kind of codebases, one is source code repositories that only contain code, and then the internet itself contains an amazing amount of code examples that are contained in blog posts and website and discussion forums, and that is selected source code, selected for its usefulness for learning and copy pasting. That is often more valuable that the one found in source repositories. So that's why these so-called code search engines are of limited use since they do not search those examples. And there is research to fix this.
- Developers prob more use refind in their local code, and more often discovery and learning on the interweb. But it can also be that parts of your local code based are unknown to you, and you could recall that you've seen something on the internet.
- Maybe you recall the song that played on the webradio, so you look that up ...
- You can use one kind of information as a proxy for other information, so you can you use social and temporal information as a proxy for dependency information (like Tom Zimmermanns work on recommending other locations to update).
- Photographic memory.
- Struggling very hard to find that one magic name, is this because of a lack of tools that support searching by other clues.
- The clues have to be turned into actions.
- How many (inter)actions do you have to do to follow up one clue.
- Non-conventional questions that people ask themselves but had no tool to answer them before.


- When developers have a technical question and are looking for the answer.
- How do developers find their way in source code? 
- temporal/episodic is often 'refind' ... social is more 'search'
- Tasks: How to go from A to B in code? Refind code. How do you find a certain piece of code? How do you implemented a certain functionality? When you do a change how do find other places to be changed? If you dont know how to do something where you learn how to do it? Where do you find code examples?
- Two hard problems in CS, caching and naming.

- Countable vs unlimited, homogenous vs heterogenous
- Search in your own system is limited and homogenous, if you search on the web you search in an unlimited and heterogenous system, where you gonna have problems like trust. If you search within company you trust, outside credibility is less given
- JExample generates 'Examples that are worth to be found' .. JExample makes exampels more relevant and more applicable ...
- Erwann: information can come right from the source code (lexical clues) and can come from somebody else (social clues) from temporal memory or version control system (temporal ones) ...
- Clues are not about the sources of information, but about the way developers navigate about how they cognitively work when they follow.

	
%- So one question is, do we observe this pattern because lexical clues are the only ones with decent tool support?
%- All these tools help you to follow up, some more complex clues and unconventional information. It is not that any of these tools addresses a very specific task? Erwann thinks yes it does.
%- I want to help developers, working with code, in particular wrt navigation and understanding, and learning about code.  These clues can be  (ii)  or (iii) it could be temporal or episodic, having a memory from the past that could lead you to an answer, that you've seen it last year (or recall the song that played on the radio) or (iv) 
%- maybe in some years we can add here location based clues, like I've written that code on the train. And in the source of JUnit Kent Beck found it so cool that he added "was written on H�tte at 1200 mUm"
%- and sometime people follow up some very underdeveloped spatial clues. So we know they follow up these clues so we need better tool support and not have to be force to use lexical proxies, you should not be forced to recall the name, but be able to directly work social information when following up a social clues.
%- With spatial information we have this even more special situation that we possible first have to create a space for space.
%- Is there one space to rule them all? or different spaces for different tasks. Codemap that provides one space but still have IDE so onscreen is split, Codecanvas does the same but put the source code on it such that the space become the only onscreen space, the space should be contextually created.

%- Spatial one, two things that are close together have something do together, if we put randomly put things on screen (not quite correct...) of course this works because people as humans have strong spatial capabilities, that software can be represented in spatial meaningful way, this is really good enough because  the spatial part of our brain that is impressive, if people had really good memory for which tunes and sound there would be one how can source code be distributed over the vocal range. Why do we not sing source code to one other. We could sing quick sort :)

%- The thesis emerged from my work, is that in order to support developers with code navigation we need to build better tools that address the cognitive clues that are used by developers, in particular (in addition to lexical clues) to support navigation by social clues by temporal clues and by spatial clues, and since software has no ...
%- lexical clues are present as identifier names and in comments, social clues are also there we know how has written software and mailinglists etc so people are also there as entity, and also temporal information is tere we have versioning systems that make snapshots of a system in time, but spatial is more difficult ...
%- it somehow maps to the structure of software, which you can extract from source code, but very often spatial is more on a conceptual level and people even use spatial terms on the very level of the arrangement of the UI on screen. So we have three spatial dimensions, structural, conceptual and on screen. So the spatial clues have the least support, so we gotta create a space of software of onscreen to support spatial navigation by developers. 
%- Since software lacks we need to create a spatial view to address the on-screen spatial clues. To create a cartographic representation of other non-spatial properties of the system (like temporal, social and lexical ones) and give them a spatial representation os people can use this forth kind of navigation clues that are actually used. When you talk with developers how they find answers to technical questions related to source code they will tell you sometime they recall a method is at the end of this file, which is onscreen spatial representation. Lemme browser the packages it us up there, down there. And they will use this in the same sentence as some spatial representation that is related to package structure. We have to fix this also up there in the business layer, it as down there in this file, we have to fix this up there it is down there in this file. 

\subsection*{Orientation by Lexical Clues}

% Importance
Lexical clues are very important for code navigation. They are by far the most common clue used by software engineers for \codenavigation. 
% Examples
Examples of lexical clues are the name of an identifier used to refind some code, or a keyword used to search the web for documentation or code examples. 

% Assumption
Developers do rely on lexical clues because they assume that names are meaningful, that there is a meaning to names and that names have been meaningfully chosen. 
% Kind of information?
Lexical clues are typically pointers to lexical information that is present as identifier names and in comments. However, lexical information is not limited to source code but also present in the content of an email, a bug report or a web page. For example, following-up a lexical clue might guide the developer to a wikipedia page that contains the algorithm he's looking for. 

% Tool support and hint a need that we address later on?
Simple keyword search and regular expressions are of great help to follow-up on a lexical clue. However, a major problem with lexical clues is that developers to have to make a guess about how other people name things. Using information retrieval and natural language processing can help to address that problem. 

% Lexical clues are very important for search. Often process that you first need to find lexical clue so you know here do gonna search, for example name of a class or method

\subsection*{Orientation by Social Clues}

% Importance
Social clues are often not considered part of software comprehension, yet they are possibly the most powerful clues for code orientation. 
% Examples
Examples of social clues can be asking another person for help, or posting a question to a discussion board on the web where experts gather and share their expertise. 

% Assumption 
Developers do rely on social clues because they assume that other people do have expertise, that these other people do have more or different expertise so they can be of help to find answers. 
% Kind of information?
Social clues are typically pointers to other persons from the developer's personal network, either a co-worker or a friend. However, social clues are not limited to the personal network but can also appear in the form of mailing lists and other expert groups that are ready to share their expertise online on the internet.

% Tool support and hint a need that we address later on?
Support for code orientation by social clues is typically not present in development tools, thus the current state of the art is that developer have to recall the name of a person with expertise. This basically boils down to a lexical clue that has to be used a proxy for the actual social clue that developer does want to follow-up. Using techniques and ideas drawn from social media as found on the web can help to address that problem.

\subsection*{Orientation by Episodic Clues}

% Importance
Episodic clues are most helpful to refind source code that has been written or used in the past, since as humans we have strong episodic memories. 
% Examples
Episodic clues are typically tied to personal memory, as for example in recalling the first-hand experience of a conference talk or a pair programming session. 

% Assumption 
Developers do rely on episodic clues because humans do have strong episodic memory and thus their guess about the past is more correct than their current knowledge. This works is because as humans our episodic memory works way better than our structural memory, typically developers do recall ``that'' they knew the answer before but not ``what'' the answer exactly constituted of.  
% kind of information?

Episodic clues are typically pointers to past interaction with books, mailing lists and other people, but may also be pointers to past snapshots of the current or a related software system. 
% tool support and hint a need that we address later on?
Support for code orientation by episodic clues in development tools is typically not present. Information that could be used to follow-up an episodic clue is typically stored in external databases, such as version control repositories and mailing list archives, and thus typically not integrated with development tools. Also it is known that visualizations are a powerful means to evoke episodic memories. Using data mining techniques and embedding the results as visualizations in the IDE can help to fill that gap.

%- or you know ''we had this bug a year ago'' or you know ''oh we need to sort a list, we've done this in this project'' ... episodic memory ... or you recall who've told you about it ... 
%- Episodic memory ... recall that they've solved it ... or that they've seen the solution on a mailing list ... learning by lurking ... grown your own folksonomy ... collect your own anecdotes
%- Chronia: ownership map shows how contributers to a software system worked together ... devs get excited ... ''see here we worked together'' .. even better than a holiday photo album.

\subsubsection{Spatial}

- Change on database propagates to business layer, UI layer ... you know where do go spatially, the space is given by space of software, in other case the space is given by layout of IDE, or given by storage format or persistence format, you know it is there in that text file or on that folder, or in smalltalk open in that window.
- Codemap: if you observe which clues devs use to refer to code, you'll find all four, but if you look at tool support, you find lexical supported by tools, and temporal only in external tools, and social clues are also not supported, you gonna recall the name of person (ie using the lexical clue of the name as a proxy for the social connection) and spatial is not supported at all, we got the IDE layouted without care about spatial thinking ... code bubbles, code map, and code canvas are about the first systems to take care of this ... 
- If you ask people to draw maps of their software systems ... as Grady Booch does ... and they always draw an architecture with spatial structural ... so the spatial structure is in their thinking but not in the IDE and even not in the software ...
- So this why codemap is so interesting because it takes the lexical information and give it a spatial structure in the hope to fill the gap when devs are stuck with lexical clues or social/temporal
- Spatial clues "somewhere at the end of this file"
- Structural and spatial is often perceived in the same way, developers tend to use spatial terminology to refer to layers and architectural components.

- To create a space of software that is visualized on screen, so that people can start using that one to navigate in source code.
- Erwann: the mental model part?
- To provide them with a space that will become / should reflect their spatial model...
- Erwann: if the notion of mental model can be used for all clues ... the other tools gravitate around this spatial mental model.


\begin{itemize}
\item \emph Simple keyword search and regular expressions over the source code are of great help to follow up a lexical clue. The true power of lexical clues, however, is only unleashed when applying information retrieval and natural language processing. 
\begin{itemize}
\item In \autoref{the chapter on lexical clues} we use lexical information found in source code to cluster and summarize software systems.
\item In \autoref{the chapter on LogLR} we use lexical information found to summarize whole systems, parts thereof, or even the system's entire evolution.
\item In \autoref{the chapter on bug reports} we show how to automatically assign work tickets using lexical information found in both source code and bug reports.
\item In \autoref{the chapter on codemap} we use lexical information found in source code to establish a visualization that supports spatial code orientation.
\end{itemize}
\item \emph{Spatial clues} can help to reduce the cognitive load of orientation of an hyperlinked document space, such as software systems. Spatial clues can be of two kinds, either they are structural as in recalling to which class a certain method belonged or true spatial clues as in recalling that a given method was at the end of a certain source file. Current support for code orientation by spatial clues in development tools is \adhoc at best, the systems is presented as a tree view of alphabetically ordered files and inside a file the source is linearized as an unordered text file. While this might accidentally help to refind some code by a spatial clue, support for the discovery of yet unknown source code by spatial clues is limited to structural clues.
%Always given the assumption that the system is well packaged, which often is not given either.
A key contribution of our research is the creation of novel spatial visualization to support code orientation by spatial clues.
\begin{itemize}
\item In \autoref{the chapter on chronia} we use temporal information taken from version control repositories to reveal the social interaction in collaborative software development.
\item In \autoref{the chapter on codemap} we use lexical information to establish a spatial visualization that supports spatial code orientation by individuals or teams.
\item In \autoref{the chapter on the codemap user study} we report on a qualitative user study that evaluates a prototype implementation of the techniques introduced in the chapter above. 
\end{itemize}

\item 
\begin{itemize}
\item In \autoref{the chapter on LogLR} we use lexical clues found in the version control system to summarize and reveal the story of system's evolution. 
\item In \autoref{the chapter on bug reports} we use lexical clues found in past contributions of developers to model their expertise for the assignment of bug reports. 
\item In \autoref{the chapter on chronia} we address the episodic memory of developers by providing them with a visualization that tells the story of a team's collaboration as recorded by the version control system.
\end{itemize}

\item \emph{Social clues} are often not considered part of software comprehension, yet they are possibly the most powerful clues for code orientation. Examples of social clues can be asking another person for help, or posting a question to a discussion board on the web where experts gather and share their expertise. Current support for code orientation by social clues is typically not present in development tools. To reveal the potential of social clues we integrated ideas taken from social media in our prototype tools.

\begin{itemize}
\item In \autoref{the chapter on bug reports} we use contributions that developers shared with open source systems in the past to build a recommendation model for bug repots.
\item In \autoref{the chapter on chronia} we use the past collaboration of the developers to tell a system's past story to both its team members as well as new hires.
\item In \autoref{the chapter on codesearch} we use cross-project collaboration in open source projects to model the credibility of code search results.
\end{itemize}

\end{itemize}

\section{Thesis Statement}

We state our thesis as follows

\begin{quote}
To support software engineers with code navigation and understanding, we need (1) tools that address all cognitive clues used by developers, in particular (2) code orientation by lexical, spatial and temporal clues. Since software has no inherent spatiality, for code orientation by spatial clues we need to (3) establish a novel spatial representation of software systems based on other properties, as \eg lexical information.
\end{quote}

- Thesis: "To support code navigation and understanding, developers need support for the cognitive clues they follow. In particular for social and temporal clues we need to go beyond the current state of tool, and third because software has no space we need to create some space so developers can start to use spatial clues."

\section{Our Solution in a Nutshell}

In this work, we present the following contributions that explore the use of orientation clues for tool building. Each of these contributions has been published as one or more peer-reviewed paper at an international venue or in an international journal.  

- Information retrieval to make more interesting things with lexical clues.
- Evoclouds combine lexical with temporal information.
- Work by Dominique Matter where we go and mine all the date from the temporal repository and mine the the vocabulary from the contribution of the developers, so we have a new tool that models the expertise of the developers so that now when you have a bug report the system can recommend to you the person. You can turn now a fuzzy textual clue into a social thing.
- So we asses the credibility of search results solely based on the social network of the authors. This is the way developers have been found to asses the trustworthiness of code, since when you copy code from the internet you do so because you do not want to make the effort of technically understanding it so you look at its author's credibility to make an assessment of trustworthiness.
- So we have the Chronia tool that has a timeline of the project, and you have a timeline of each file and you see the collaboration patterns. When you show this to developers the go crazy, with this visualization the episodic memory is brought back. You can learn about team members.


\begin{itemize}
\item Software Clustering \cite{Kuhn07a,Kuhn05a,Lung05a}
\item  Feature Classification \cite{Kuhn06c,Kuhn05b}
\item  Software Summarization \cite{Kuhn09a}
\item  Spatial Representation \cite{Kuhn10c,Kuhn10b,Duca06c,Kuhn08a}
\item  Ownership Map \cite{Girb05a}
\item  Bug-Report Triage \cite{Matt09a}
\item  Credibility in Code Search \cite{Gysi10b}
\end{itemize}

\section{Contributions}

%%%

Acquiring knowledge about a software system is one of the main activities in software reengineering, it is estimated that up to 60 percent of software maintenance is spent on comprehension \cite{Abra04a}. This is because a lot of knowledge about the software system and its associated business domain is not captured in an explicit form. Most approaches that have been developed focus on program structure \cite{Duca05b} or on external documentation \cite{Maar91a,Anto02b}. However, there is another fundamental source of information: the developer knowledge contained in identifier names and source code comments.

{\small\begin{quotation}\emph{The informal linguistic information that the software engineer deals with is not simply supplemental information that can
be ignored because automated tools do not use it. Rather, this information is fundamental. [\ldots] If we are to use this informal information in design recovery tools, we must propose a form for it, suggest how that form relates to the formal information captured in program source code or in formal specifications, and propose a set of operations on these structures that implements the design recovery process} \cite{Bigg89c}.
\end{quotation}}

Languages are a means of communication, and programming languages are no different. Source code contains two levels of communication: human-machine communication through program instructions, and human to human communications through names of identifiers and comments. Let us consider a small code example:

When we strip away all identifiers and comments, from the machine point of view the functionality remains the same, but for a human reader the meaning is obfuscated and almost impossible to figure out. In our example, retaining formal information only yields:

When we keep only the informal information, the purpose of the code is still recognizable. In our example, retaining only the naming yields:

is morning hours minutes seconds is date hours minutes
seconds invalid time value hours 12 minutes 60 seconds 60

\section{Outline}

The dissertation is structured as follows

\begin{enumerate}
% --> related.tex
\item[\autoref{the chapter on related work}] discusses the related work of this thesis. We present various user studies and solutions to code orientation and analyse the short comings in the context of each orientation clue.
% --> codemap.tex
\item[\autoref{the chapter on codemap}] we introduce SOFTWARE CARTOGRAPHY and approach that uses lexical information to establish a cartographic visualization that supports spatial code orientation by individuals or teams. The approach is implemented in the CODEMAP tool.
\item[\autoref{the chapter on the codemap user study}] reports on a qualitative user study that evaluates the prototype implementation of the techniques introduced in the chapter above.]
\item[\autoref{the chapter on the MSR user study}] presents a user study that looked at how developers find answers to technical questions and discusses code orientation by cognitive clues.
% --> gutter+stars.tex
\item[\autoref{the chapter on lexical clues}] presents an approach to cluster and summarize software systems using lexical information found in source code. The approach is implemented in the HAPAX tool. 
\item[\autoref{the chapter on LogLR}] presents an approach that uses lexical information found in source code to summarize whole systems, parts thereof, or even the system�s entire evolution. The approach is implemented in the EVOCLOUD tool.
\item[\autoref{the chapter on chronia}] presents an approach to address the episodic memory of developers by providing them with a visualization that tells the story of a team's collaboration as recorded by the version control system. The approach is implemented in the CHRONIA tool.
\item[\autoref{the chapter on bug reports}] presents an approach that use lexical information found in contributions that developers shared with open source systems to build a recommendation model for bug repots. The approach is implemented  in the DEVLECT tool.
\item[\autoref{the chapter on codesearch}] presents an approach that uses cross-project collaboration of developers in open source projects to model the credibility of code search results. The approach is implemented in the BENDER tool.
\item[\autoref{the conclusion}] concludes the dissertation and outlines future work.
\end{enumerate}


%%%%%%%%%%%%

